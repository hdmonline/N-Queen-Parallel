\documentclass[twoside,12pt]{article}
\usepackage{amsmath,amsfonts,amsthm,fullpage}
%\usepackage{algorithm}
\usepackage{algorithmic}
\usepackage{float}
\usepackage{graphicx}
\usepackage{listings}
\usepackage[utf8]{inputenc}
\usepackage[vlined, ruled, boxed]{algorithm2e}
\title{Programming Assignment 2}
\author{Han Dongmin, Yuanlai Zhou, Chong Ye}

\begin{document}

\maketitle

\section{Preamble}
For question 3, I discussed with Dongmin Han and Yuanlai Zhou and we worked together to figure out the parallel solution. 


\section{Algorithm}
\begin{enumerate}
\item
\textbf{seq\_solve}r
\begin{enumerate}
	\item $ op(a,b)=2a+b $ \\
	since \[ op(op(a,b),c)=2(2a+b)+c=4a+2b+c \]
	\[ op(a,op(b,c))=2a+(2b+c)=2a+2b+c \]
	\[ \Rightarrow op(op(a,b),c) \neq op(a,op(b,c)) \]
	Thus $ op(a,b)=2a+b $ is not associative operation.Therefore, parallel prefix algorithm cannot be used in this binary operation.
	\item $ op(a,b)=\sqrt{a^2+b^2} $\\
	Since
	\[ op(op(a,b),c)=\sqrt{(\sqrt{a^2+b^2})^2+c^2}=\sqrt{a^2+b^2+c^2} \]
	\[ op(a,op(b,c))=\sqrt{a^2+(\sqrt{b^2+c^2})^2}=\sqrt{a^2+b^2+c^2} \]
	\[ \Rightarrow op(op(a,b),c)= op(a,op(b,c))\]
	Thus $ op(a,b)=\sqrt{a^2+b^2} $ is associative operation.Therefore, parallel prefix algorithm can be used in this binary operation.
\end{enumerate}

\setlength{\parindent}{1em}
\item \textbf{nqueen\_master}


\vspace{3mm}

\item \textbf{nqueen\_worker} \\

\end{enumerate}

\section{optimization}

\section{Implementation and Analysis}
\begin{enumerate}
\item vary n
choose  k $ \geq 8 $
\item vary p
n not too low or too high
\item vary k
$ n>10 $
\end{enumerate}

\section{Results and Conclustion}
     
\end{document}